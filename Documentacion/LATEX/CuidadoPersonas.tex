% CLASE
\documentclass[a4paper,12pt]{article}

% PREÁMBULO
% Paquetes:
\usepackage[spanish]{babel}
\usepackage{graphicx}

% CUERPO
\begin{document}

% Caratula
    \begin{titlepage}
        \centering
        \includegraphics[width=0.2\textwidth]{Imagenes/LogoUTN.png}\par
        \vfill
        {\bfseries\LARGE Universidad Tecnológica Nacional \par}
        \vfill
        {\scshape\Large Facultad Regional Rosario \par}
        \vfill
        {\scshape\Huge Aplicación de Apoyo al Cuidado de Personas \par}
        \vfill
        {\itshape\Large Ingeniería en Sistemas de Información \par}
        {\itshape\Large Proyecto Final de Carrera \par}
        \vfill
        {\Large Autores: \par}
        {\Large Carignani, Esteban Agustín \par}
        {\Large Culich, María Agustina \par}
        {\Large Montisori, Agustín \par}
        \vfill
        {\Large Tutor: \par}
        {\Large Stortoni, Silvia \par}
        \vfill
        {\Large Abril 2024 \par}
    \end{titlepage}

% Abstract
    \begin{abstract}
        Cuidar de otra persona, ya sea un adulto mayor, 
        un niño, una persona con discapacidad o alguien 
        con una enfermedad, puede ser una tarea compleja. 
        La gestión de medicamentos, horarios, cuidadores, 
        turnos, recordatorios médicos y la comunicación 
        entre las personas involucradas.
        Basándonos en las respuestas de una encuesta realizada 
        enfocada a personas que tienen a su cargo el 
        cuidado de otras, presentamos una aplicación móvil 
        innovadora que facilita la gestión y organización de 
        las tareas que implica estar al cuidado de alguien más. 
        Nuestra propuesta es una app que permitiría el registro 
        de información personal de la persona bajo cuidado y 
        la asociación de esta al cuidador, centralización de 
        la comunicación entre ambas partes y terceros 
        involucrados, simplicidad en la organización de visitas 
        y turnos médicos, agenda de medicación, tareas 
        cotidianas, etc. Además, se incluirán herramientas que 
        respondan a las necesidades relevadas en el 
        cuestionario de sondeo.
    \end{abstract}

    
\end{document}